\documentclass[a4paper,11pt,titlepage]{article}
\usepackage[utf8]{inputenc}
\usepackage{amssymb,amsmath,amsfonts,mathrsfs}
\usepackage{enumerate,graphicx,fancyhdr,hyperref}
\usepackage[ruled]{algorithm2e}
%\usepackage{a4wide,wasysym} 
\usepackage{tikz,tkz-berge}
\usetikzlibrary{arrows,topaths,shadows}
\usepackage{subcaption}
\usepackage{wrapfig}

\tikzset{
  frame/.style={
    rectangle, draw, 
    text width=6em, text centered,
    minimum height=2em,drop shadow,fill=gray!10,
    rounded corners,
  },
  line/.style={
    draw, -latex',rounded corners=3mm,
  }
}


\hypersetup{pdfborder={0 0 0}} % links wont be visualized in the pdf

\graphicspath{{../../../figures/pdf/}{../../../figures/png/}}
\pagestyle{fancy}

% ---------------------------- Page Headers (& footers)------------------------
\lhead{Machine Intelligence I}
\fancyhead[R]{\nouppercase{\rightmark}}

% --------------------- Equation Numbering-------------------------------------
\renewcommand {\theequation}{\thesection.\arabic{equation}}			
%\newcommand{\slideref}[1]{$\Box$\footnote{slide: #1}}
\newcommand{\slideref}[1]{\marginpar{$\Box$\parbox{1.25cm}{\centering\tiny #1}}}

% ------------------------- New Commands --------------------------------------
\newcommand{\eqexcl}{\ensuremath{\stackrel{\mathrm{!}}{=}}}
\newcommand{\corresponds}{\ensuremath{\widehat{=}}}
\newcommand{\coloneqq}{\ensuremath{:=}}
\newcommand{\argmin}{\operatornamewithlimits{argmin}}
\newcommand{\sign}{\operatorname{sign}}
\newcommand{\emp}{\operatorname{emp}}
\newcommand{\dvc}{\mathrm{d}_{\mathrm{VC}}}
\newcommand{\argmax}{\operatornamewithlimits{argmax}}
\newcommand{\itr}{\item[$\rightarrow$]}
\newcommand{\itR}{\item[$\Rightarrow$]}
\newcommand{\itl}{\item[$\leadsto$]}
\renewcommand{\labelitemi}{$\blacksquare$}
\renewcommand\labelitemii{$\bullet$}
\newcommand{\iitem}[1]{\begin{itemize} \item {#1} \end{itemize}}
\newcommand{\algfont}[1]{\texttt{#1}}
% \smallfrac is a \frac in \textstyle
\newcommand{\smallfrac}[2]{ {\textstyle \frac{#1}{#2}} }
% 			--------------------
% \smallsum creates a sum in textstyle with small indices
\newcommand{\smallsum}[2]{ 
	{\textstyle \sum\limits_{\scriptscriptstyle #1}^{\scriptscriptstyle #2}} }

% matrices with different types of brackets 
\newcommand{\mat}[2][rrrrrrrrrrrrrrrrrrrrrrrrrr]{
  \left[ \begin{array}{#1} #2\\ \end{array} \right]}
\newcommand{\rmat}[2][rrrrrrrrrrrrrrrrrrrrrrrrrr]{
  \left( \begin{array}{#1} #2\\ \end{array} \right)}
\newcommand{\dmat}[2][rrrrrrrrrrrrrrrrrrrrrrrrrr]{
  \left|\left[ \begin{array}{#1} #2\\ \end{array} \right]\right|}
\newcommand{\drmat}[2][rrrrrrrrrrrrrrrrrrrrrrrrrr]{
  \left|\left( \begin{array}{#1} #2\\ \end{array} \right)\right|}

\renewcommand{\vec}[1]{\ensuremath{\underline{\mathbf{#1}}}}
% NOTE: Vector notation used in neural network textbooks -- this is 
% easier to read for expressions like \hat{\underline{\mathbf{w}}}. 

% Mathematical standard sets
\newcommand{\R}{\mathbb{R}}
\newcommand{\N}{\mathbb{N}}
\newcommand{\E}{\mathbb{E}}
\newcommand{\Set}[1]{\mathcal{#1}}